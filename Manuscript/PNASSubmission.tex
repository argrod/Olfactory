\documentclass[9pt,twocolumn,twoside,lineno]{pnas-new}
% Use the lineno option to display guide line numbers if required.
\usepackage{helvet,lettrine,minifp,mdframed,zref-abspage,needspace,authblk,xifthen,ifmtarg,colortbl,booktabs,algorithm,float,algpseudocode,changepage,newfloat,lastpage,lineno,footmisc,enumitem,sidecap,marginnote,mdframed}
\templatetype{pnasresearcharticle} % Choose template 
% {pnasresearcharticle} = Template for a two-column research article
% {pnasmathematics} %= Template for a one-column mathematics article
% {pnasinvited} %= Template for a PNAS invited submission

\title{Template for preparing your research report submission to PNAS using Overleaf}

% Use letters for affiliations, numbers to show equal authorship (if applicable) and to indicate the corresponding author
\author[a,c,1]{Author One}
\author[b,1,2]{Author Two} 
\author[a]{Author Three}

\affil[a]{Affiliation One}
\affil[b]{Affiliation Two}
\affil[c]{Affiliation Three}

% Please give the surname of the lead author for the running footer
\leadauthor{Lead author last name} 

% Please add a significance statement to explain the relevance of your work
\significancestatement{Authors must submit a 120-word maximum statement about the significance of their research paper written at a level understandable to an undergraduate educated scientist outside their field of speciality. The primary goal of the significance statement is to explain the relevance of the work in broad context to a broad readership. The significance statement appears in the paper itself and is required for all research papers.}

% Please include corresponding author, author contribution and author declaration information
\authorcontributions{Please provide details of author contributions here.}
\authordeclaration{Please declare any competing interests here.}
\equalauthors{\textsuperscript{1}A.O.(Author One) contributed equally to this work with A.T. (Author Two) (remove if not applicable).}
\correspondingauthor{\textsuperscript{2}To whom correspondence should be addressed. E-mail: author.two\@email.com}

% At least three keywords are required at submission. Please provide three to five keywords, separated by the pipe symbol.
\keywords{Keyword 1 $|$ Keyword 2 $|$ Keyword 3 $|$ ...} 

\begin{abstract}
Please provide an abstract of no more than 250 words in a single paragraph. Abstracts should explain to the general reader the major contributions of the article. References in the abstract must be cited in full within the abstract itself and cited in the text.
\end{abstract}

\dates{This manuscript was compiled on \today}
\doi{\url{www.pnas.org/cgi/doi/10.1073/pnas.XXXXXXXXXX}}

\begin{document}

\maketitle
\thispagestyle{firststyle}
\ifthenelse{\boolean{shortarticle}}{\ifthenelse{\boolean{singlecolumn}}{\abscontentformatted}{\abscontent}}{}

% If your first paragraph (i.e. with the \dropcap) contains a list environment (quote, quotation, theorem, definition, enumerate, itemize...), the line after the list may have some extra indentation. If this is the case, add \parshape=0 to the end of the list environment.
\dropcap{T}his PNAS journal template is provided to help you write your work in the correct journal format. Instructions for use are provided below. 

Note: please start your introduction without including the word ``Introduction'' as a section heading (except for math articles in the Physical Sciences section); this heading is implied in the first paragraphs. 

The marine environment inhabited by seabirds is patchy in its presence of viable prey. The distribution of plankton and large marine fauna which seabirds prey on vary both spatially and temporally, typically with smaller high density patches found within larger and less dense prey concentrations. As such, a vital aspect of seabird foraging behaviour is search, during which birds must use information gathered from their external environments to best identify where they are most likely to find food. Given the particularly large distances across which seabirds travel to foraging sites, the accuracy of their headings is hugely important. Maintenance of travel headings is important for the same reason, and provides an interesting question as to how seabirds keep a constant bearing while travelling over featureless ocean. A plausible answer lies in their olfactory capabilities.

Seabirds are highly adapted ocean inhabitants, spending upward of 90\% of their lives at sea \citep{Balance_2019}, reflecting the time spent travelling to and from foraging locations. Among those adaptations is the means to navigate the oceans and find prey across an environment sparse in visual features. Debate is ongoing over the means by which seabirds are capable of identifying and navigate towards foraging points. 

As such, foraging efficiency is largely determined by their ability to travel to a prey patch due to the large distances they must often travel \citep{Weimerskirch_2003}. How seabirds find prey has formed much debate in the scientific literature. Seabirds have developed a number of foraging methods, both in prey capture techniques and search behaviours \citep{Shealer_2001}. This variety in behaviours is reflected in the range of scientific literature available on the topic of seabird foraging. However, details on seabird foraging are as yet relatively scarce. 

\section{The senses of a seabird}
Avian travel has long held interest concerning the ability of birds to correctly and efficiently decide on travel headings to reach intended destinations. Efficient foraging requires seabirds to identify productive areas across the featureless oceans they travel. Seabirds must travel long distances, both during migrations and foraging trips. Therefore, directing travel is vital in optimising foraging efficiency, and so understanding their means of navigation during foraging trips is paramount.

% VISUAL
\subsection{Visual senses}
%VISUAL VS OTHER SENSES
Visual cues are effective methods by which information of local conditions (prey presence) can be passed on between individuals. This 'local enhancement' exchanges information of successful foraging spots. The form of these visual cues are largely made up of conspecific or other seabirds rafting on the water surface \citep{Weimerskirch_2010b}, or associations with marine predators that are well documented in the scientific literature \citep{Nevitt_1999, Sakamoto_2009a, Silverman_2004, Thiebot_2012}. These associations indicate that the visual senses are important to these animals in choosing foraging spots, as presence of these visual cues indicate a high likelihood of prey, and so potential energy gain. Similarly, use of visual guides to direct birds to nesting grounds has been suggested in previous studies, using coastlines as visible markers to follow \citep{Pollonara_2015,Yamamoto_2008}.

% SOCIAL
\subsection{Social transfer and memory}
Visual cues and local enhancement are restricted by the visual range of the searching individual. At greater distances, it is unlikely that visual cues are sufficient to guide seabirds to prey patches, especially when flying over a potentially hazy environment which may be concealed by cloud cover. The relatively direct tracks recorded of seabirds travelling to foraging spots or homing \citep{Weimerskirch_2007b,Pollonara_2015} indicate a strong navigation system involved in directing travel headings. Seabirds and large marine predators can associate known areas with persistent prey presence \citep{Davoren_2013}, and so an initial heading may be garnered from past trips. However, following this initial heading, birds must rely on some means to navigate towards likely foraging spots, particularly as they come into closer proximity to prey patches. Sharing information across conspecifics would optimise foraging trips through directing travel to known productive foraging spots. The "information center hypothesis" (ICH) theorises that such information is spread at the returning site of central place foragers, i.e. nest colonies. Such transfer of information passes on knowledge of locations of previous foraging spots, beyond visual ranges of the colonies. However, little evidence of ICH as a mechanism for information transfer at scales greater than those of visual cues exists \citep{Davoren_2003}.

% GEOMAGNETIC
\subsection{Avian geomagnetism senses}
Birds are reported to sense the Earth's geomagnetic field through receptors in their eyes and use it to orient and navigate over long distances, particularly during migration \citep{Wiltschko_2011,Hiscock_2016}. While this research has focussed on small, primarily terrestrial passerines, some experiments on seabirds have explored the relative use of geomagnetic senses in navigation. Magnets attached to Scopoli's shearwaters (\textit{Calonectris diomedea}) performing foraging trips showed no difference in navigation ability in comparison to control birds, suggesting the role of geomagnetism during such trips is reduced in comparison to other sensory input.

% OLFACTORY
\subsection{Avian olfactory senses}
% SEABIRD HEADINGS AND FINDING PREY
Homing pigeons (\textit{Columba livia}) have gained reputations as skilled navigators, capable of returning to a home site regardless of their release point. Debate about the means by which pigeons direct their travel began a line of avian research surrounding the use of atmospheric odours. Nerve sectioning to induce anosmia in pigeons was one of the initial indications of the reliance of birds on olfaction to navigate, as anosmic birds' orientations were greatly distributed \citep{Papi_1972}. The concept of an 'odour map' was conceived and developed as a way for birds to orient themselves based on odour characteristics of their environment \citep{Papi_1972,Bonadonna_2003}. Such discussion has focussed on seabirds, wide-ranging animals that primarily travel over a featureless landscape. Tube-nosed seabirds (of the order Procellariiformes) in particular have received attention due to their enhanced olfactory apparatus \citep{Bang_1960, Bang_1971, Jacobs_2012}. The development of theories regarding use of olfaction in seabirds began to spread across the scientific community following studies involving Procellariiform seabirds displaying attraction to artificial oil slicks \citep{Grubb_1972,Dell_Ariccia_2014,Nevitt_1995,Nevitt_1999,Nevitt_2000}. These studies showed attraction of Procellariiformes to artificial concentrations of cod liver oil and/or dimethyl sulphide (DMS). DMS is a naturally occurring by-product of decomposition of dimethylsulphoniopropionate, produced when zooplankton graze on phytoplankton. DMS is therefore an indicator of zooplankton presence, ergo indicative of high productivity \citep{Cantin_1996,Dacey_1986,Jean_2009,Sim__2001}. Abilities of Procellariiformes to detect DMS has similarly been shown through experimental testing. Blue petrel chicks (\textit{Halobaena caerulea}) displayed sensitivity to picomolar concentrations of DMS in controlled experiments \citep{Bonadonna_2006}.




Far-ranging seabirds travel over visibly featureless oceans during foraging trips. This has lead to 
% The study of marine ecology has long been a source of interest for researchers given the widely unknown oceanic environment. Humans rely on the oceans as a source of food, 
% and more recently as a great feedback system for many environmental factors. As such, sampling the oceans is of great benefit to researchers studying the status of the climate of the Earth. 
% The foraging ecology of seabirds is a vital aspect of their 
Foraging behaviour can be separated into distinct phases, search and prey capture. Both these components of foraging are important in reconstructing the ecology of a species and understanding how their foraging behaviour may change in response to changes to its environment and/or prey abundance. Seabirds provide a unique case of intrigue for researchers due to their wide-ranging nature and the environment which they inhabit. How these animals find prey is a question that has formed much debate in the scientific literature.

% VISUAL
\subsection{Procellariiform visual sense}
%VISUAL VS OTHER SENSES
Efficient foraging requires seabirds to identifty productive areas across the featureless oceans they travel. Visual cues are effective methods by which information of local conditions (prey presence) can be passed on between individuals. This local enhancement exchanges information of successful foraging spots. The form of these visual cues can be made up of conspecific or other seabirds rafting on the water surface \citep{Weimerskirch_2010b}, or associations with marine predators that are well documented in the scientific literature \citep{Nevitt_1999, Sakamoto_2009a, Silverman_2004, Thiebot_2012}. These associations indicate that the visual senses are important to these animals in choosing foraging spots, as presence of these visual cues indicate a high likelihood of prey, and so potential energy gain. Similarly, use of visual guides to direct birds to nesting grounds has been suggested in previous studies, using coastlines as visible markers to follow \citep{Pollonara_2015,Yamamoto_2008}.

% SOCIAL
\subsection{Social transfer and memory}
Visual cues and local enhancement are restricted by the visual range of the searching individual. At greater distances, it is unlikely that visual cues are sufficient to guide seabirds to prey patches, especially so when flying over a potentially hazy environment which may be concealed by cloud cover. The relatively direct tracks recorded of seabirds travelling to foraging spots or homing \citep{Weimerskirch_2007b,Pollonara_2015} indicate a strong navigation system involved in directing travel headings. Seabirds and large marine predators can associate known areas with persistent prey presence \citep{Davoren_2013}, and so an initial heading may be garnered from past experience. However, following this initial heading, birds must rely on some means to navigate towards likely foraging spots, particularly as they near foraging spots.

% OLFACTORY
\section{Avian olfactory senses}
% SEABIRD HEADINGS AND FINDING PREY
Avian travel has long held interest concerning the ability of birds to correctly decide on travel headings to not only reach intended destinations, but to do so efficiently. Homing pigeons (\textit{Columba livia}) have gained reputations as canny navigators, capable of returning to a home site regardless of their release point. Debate about the means by which pigeons direct their travel began a line of avian research surrounding the use of atmospheric odours. Nerve sectioning to bring about anosmia in pigeons was one of the initial indications of the reliance of birds on olfaction to navigate \citep{Papi_1972}. The concept of an 'odour map' was conceived and developed as a way for birds to orient themselves based on odour characteristics of their environment \citep{Papi_1972,Bonadonna_2003}. Such discussion has focussed on seabirds, wide-ranging animals that primarily travel over a featureless landscape. Tube-nosed seabirds (of the order Procellariiformes) in particular have received attention due to their enhanced olfactory apparatus \citep{Bang_1960, Bang_1971, Jacobs_2012}. The development of theories regarding use of olfaction in seabirds began to spread across the scientific community following studies involving Procellariiform seabirds displaying attraction to artificial oil slicks \citep{Grubb_1972,Dell_Ariccia_2014,Nevitt_1995,Nevitt_1999,Nevitt_2000}. These studies showed attraction of Procellariiformes to articifial concentrates of cod liver oil and/or dimethyl sulphide (DMS). DMS is a naturally occurring by-product of decomposition of dimethylsulphoniopropionate, produced when zookplanton graze on phytoplankton. DMS is therefore an indicator of zooplankton presence and so areas of high productivity \citep{Cantin_1996,Dacey_1986,Jean_2009,Sim__2001}. Abilities of Procellariiformes to detect DMS has similarly been shown through experimental testing. Blue petrel chicks have shown sensitivity to picomolar concentrations of DMS in controlled experiments \citep{Bonadonna_2006}.

% olfactory map

Sharing information across conspecifics would optimise foraging trips through directing travel to known productive foraging spots. The "information center hypothesis" (ICH) theorises that such information is spread at the returning site of central place foragers, i.e. nest colonies. Such transfer of information passes on knowledge of locations of previous foraging spots, beyond visual ranges of the colonies. Little evidence of ICH as a mechanism for information transfer at scales greater than those of visual cues exists \citep{Davoren_2003} 

% BIOLOGGING
\section{Studying seabirds through biologging}
% BIOLOGGING HISTORY
Studying behaviours of marine species has inherent difficulties in observing these species both above and, more importantly, below the water surface. The field of biologging, where tags recording data attached to animals enable collection of information on their movements during natural behaviours, provided an innovative solution to this challenge. This information can then be used to infer behaviour or explore the effects of external factors of study species. Since the inception of biologging, using capillary tubes and ink to measure maximum pressures experienced by seals during dives \citep{Kooyman_1965}, development of smaller and more sophisticated tags has expanded the possibilities of biologging studies through recording of novel data types and the use of innovative analytical methods \citep{Kooyman_2004}. In biologging studies, ensuring tagged individuals do not experience any reduction of fitness or ability to perform typical behaviours is paramount. This is a key factor when conceiving a biologging-based study and has proved to be a considerable challenge in previous research \citep{Gessaman_1988, Bowlin_2010}. As flight and flight efficiency is of utmost importance to seabirds, tag mass is a key factor in ensuring this lack of unintended impact during tag deployments. As a result, tagging of seabirds is often limited by the relative mass of tags to the tagged individual's body mass.

% NOVEL ASPECTS
Thus far, studies on the effects of wind and/or the the sensory capabilities of foraging seabirds have relied on satellite data gathered through tools such as the Advanced SCATterometer (ASCAT) \citep{Bentamy2012}. However, these data differ both temporally and spatially from the changes in foraging routes performed by seabirds. Recent studies have developed means to estimate wind conditions from seabird GPS tracks, using the relationship between ground speeds and heading \citep{Yonehara_2016}, and the relationship of the bird headings to the mean track vector \citep{Goto_2017}.

In combination with estimates of foraging method, we can investigate the search behaviour as the animals travel towards their foraging locations.  

\section*{Guide to using this template on Overleaf}

Please note that whilst this template provides a preview of the typeset manuscript for submission, to help in this preparation, it will not necessarily be the final publication layout. For more detailed information please see the \href{https://www.pnas.org/page/authors/format}{PNAS Information for Authors}.

If you have a question while using this template on Overleaf, please use the help menu (``?'') on the top bar to search for \href{https://www.overleaf.com/help}{help and tutorials}. You can also \href{https://www.overleaf.com/contact}{contact the Overleaf support team} at any time with specific questions about your manuscript or feedback on the template.

\subsection*{Author Affiliations}

Include department, institution, and complete address, with the ZIP/postal code, for each author. Use lower case letters to match authors with institutions, as shown in the example. PNAS strongly encourages authors to supply an \href{https://orcid.org/}{ORCID identifier} for each author. Individual authors must link their ORCID account to their PNAS account at \href{http://www.pnascentral.org/}{www.pnascentral.org}. For proper authentication, authors must provide their ORCID at submission and are not permitted to add ORCIDs on proofs.

\subsection*{Submitting Manuscripts}

All authors must submit their articles at \href{http://www.pnascentral.org/cgi-bin/main.plex}{PNAScentral}. If you are using Overleaf to write your article, you can use the ``Submit to PNAS'' option in the top bar of the editor window. 

\subsection*{Format}

Many authors find it useful to organize their manuscripts with the following order of sections;  title, author line and affiliations, keywords, abstract, significance statement, introduction, results, discussion, materials and methods, acknowledgments, and references. Other orders and headings are permitted.

\subsection*{Manuscript Length}

A standard 6-page article is approximately 4,000 words, 50 references, and 4 medium-size graphical elements (i.e., figures and tables). The preferred length of articles remains at 6 pages, but PNAS will allow articles up to a maximum of 12 pages.

\subsection*{References}

References should be cited in numerical order as they appear in text; this will be done automatically via bibtex, e.g. \cite{belkin2002using} and \cite{berard1994embedding,coifman2005geometric}. All references cited in the main text should be included in the main manuscript file.

\subsection*{Data Archival}

PNAS must be able to archive the data essential to a published article. Where such archiving is not possible, deposition of data in public databases, such as GenBank, ArrayExpress, Protein Data Bank, Unidata, and others outlined in the \href{https://www.pnas.org/page/authors/journal-policies#xi}{Information for Authors}, is acceptable.

\subsection*{Language-Editing Services}
Prior to submission, authors who believe their manuscripts would benefit from professional editing are encouraged to use a language-editing service (see list at www.pnas.org/page/authors/language-editing). PNAS does not take responsibility for or endorse these services, and their use has no bearing on acceptance of a manuscript for publication. 

\begin{figure}%[tbhp]
\centering
\includegraphics[width=.8\linewidth]{frog}
\caption{Placeholder image of a frog with a long example legend to show justification setting.}
\label{fig:frog}
\end{figure}


\begin{SCfigure*}[\sidecaptionrelwidth][t]
\centering
\includegraphics[width=11.4cm,height=11.4cm]{frog}
\caption{This legend would be placed at the side of the figure, rather than below it.}\label{fig:side}
\end{SCfigure*}

\subsection*{Digital Figures}

EPS, high-resolution PDF, and PowerPoint are preferred formats for figures that will be used in the main manuscript. Authors may submit PRC or U3D files for 3D images; these must be accompanied by 2D representations in TIFF, EPS, or high-resolution PDF format. Color images must be in RGB (red, green, blue) mode. Include the font files for any text.

Images must be provided at final size, preferably 1 column width (8.7cm). Figures wider than 1 column should be sized to 11.4cm or 17.8cm wide. Numbers, letters, and symbols should be no smaller than 6 points (2mm) and no larger than 12 points (6mm) after reduction and must be consistent. 

Figures and tables should be labelled and referenced in the standard way using the \verb|\label{}| and \verb|\ref{}| commands.

Figure \ref{fig:frog} shows an example of how to insert a column-wide figure. To insert a figure wider than one column, please use the \verb|\begin{figure*}...\end{figure*}| environment. Figures wider than one column should be sized to 11.4 cm or 17.8 cm wide. Use \verb|\begin{SCfigure*}...\end{SCfigure*}| for a wide figure with side legends.

\subsection*{Tables}
Tables should be included in the main manuscript file and should not be uploaded separately.

\subsection*{Single column equations}

Authors may use 1- or 2-column equations in their article, according to their preference.

To allow an equation to span both columns, use the \verb|\begin{figure*}...\end{figure*}| environment mentioned above for figures.

Note that the use of the \verb|widetext| environment for equations is not recommended, and should not be used. 

\begin{figure*}[bt!]
\begin{align*}
(x+y)^3&=(x+y)(x+y)^2\\
       &=(x+y)(x^2+2xy+y^2) \numberthis \label{eqn:example} \\
       &=x^3+3x^2y+3xy^3+x^3. 
\end{align*}
\end{figure*}


\begin{table}%[tbhp]
\centering
\caption{Comparison of the fitted potential energy surfaces and ab initio benchmark electronic energy calculations}
\begin{tabular}{lrrr}
Species & CBS & CV & G3 \\
\midrule
1. Acetaldehyde & 0.0 & 0.0 & 0.0 \\
2. Vinyl alcohol & 9.1 & 9.6 & 13.5 \\
3. Hydroxyethylidene & 50.8 & 51.2 & 54.0\\
\bottomrule
\end{tabular}

\addtabletext{nomenclature for the TSs refers to the numbered species in the table.}
\end{table}

\subsection*{Supporting Information Appendix (SI)}

Authors should submit SI as a single separate SI Appendix PDF file, combining all text, figures, tables, movie legends, and SI references. SI will be published as provided by the authors; it will not be edited or composed. Additional details can be found in the \href{https://www.pnas.org/authors/submitting-your-manuscript#manuscript-formatting-guidelines}{PNAS Author Center}. The PNAS Overleaf SI template can be found \href{https://www.overleaf.com/latex/templates/pnas-template-for-supplementary-information/wqfsfqwyjtsd}{here}. Refer to the SI Appendix in the manuscript at an appropriate point in the text. Number supporting figures and tables starting with S1, S2, etc.

Authors who place detailed materials and methods in an SI Appendix must provide sufficient detail in the main text methods to enable a reader to follow the logic of the procedures and results and also must reference the SI methods. If a paper is fundamentally a study of a new method or technique, then the methods must be described completely in the main text.

\subsubsection*{SI Datasets} 

Supply .xlsx, .csv, .txt, .rtf, or .pdf files. This file type will be published in raw format and will not be edited or composed.


\subsubsection*{SI Movies}

Supply Audio Video Interleave (avi), Quicktime (mov), Windows Media (wmv), animated GIF (gif), or MPEG files. Movie legends should be included in the SI Appendix file. All movies should be submitted at the desired reproduction size and length. Movies should be no more than 10MB in size.


\subsubsection*{3D Figures}

Supply a composable U3D or PRC file so that it may be edited and composed. Authors may submit a PDF file but please note it will be published in raw format and will not be edited or composed.


\matmethods{Please describe your materials and methods here. This can be more than one paragraph, and may contain subsections and equations as required. 

\subsection*{Subsection for Method}
Example text for subsection.
}

\showmatmethods{} % Display the Materials and Methods section

\acknow{Please include your acknowledgments here, set in a single paragraph. Please do not include any acknowledgments in the Supporting Information, or anywhere else in the manuscript.}

\showacknow{} % Display the acknowledgments section

% Bibliography
\bibliography{pnas-sample}

\end{document}
